%!TEX TS-program = xelatex
%!TEX encoding = UTF-8 Unicode
%----------------------------------------------------------------
% 配置
%----------------------------------------------------------------
% A4纸张大小默认情况下,'letterpaper'使用信纸
\documentclass[11pt, a4paper,AutoFakeBold]{awesome-cv}

% 使用geometry配置页边距
\geometry{left=1.4cm, top=.8cm, right=1.4cm, bottom=1.8cm, footskip=.5cm}

% 指定包含字体的位置
\fontdir[fonts/]

% 高光颜色
% Awesome Colors: awesome-emerald, awesome-skyblue, awesome-red, awesome-pink, awesome-orange
%                 awesome-nephritis, awesome-concrete, awesome-darknight
\colorlet{awesome}{awesome-red}
% 如果要指定自己的颜色,请取消注释
% \definecolor{awesome}{HTML}{CA63A8}

% 文本的颜色
% 如果要指定自己的颜色,请取消注释
% \definecolor{darktext}{HTML}{414141}
% \definecolor{text}{HTML}{333333}
% \definecolor{graytext}{HTML}{5D5D5D}
% \definecolor{lighttext}{HTML}{999999}

% 如果您不想突出显示具有令人敬畏的颜色的部分,请设置为false
\setbool{acvSectionColorHighlight}{true}

% 如果要将社交信息分隔符从“|”更改为其他内容
\renewcommand{\acvHeaderSocialSep}{\quad\textbar\quad}


%----------------------------------------------------------------
%	个人信息
%	如果不需要,请注释下面的任何一行
%----------------------------------------------------------------
% 可用选项: circle|rectangle,edge/noedge,left/right
\photo[rectangle,edge,right]{./examples/profile}
\name{姓}{名}
\position{软件架构师{\enskip\cdotp\enskip}安全专家}
\address{英国伦敦贝克街221号}

\mobile{(+86) 166-6666-6666}
\email{posquit0.bj@gmail.com}
\homepage{www.posquit0.com}
\github{posquit0}
\linkedin{posquit0}
% \gitlab{gitlab-id}
% \stackoverflow{SO-id}{SO-name}
% \twitter{@twit}
% \skype{skype-id}
% \reddit{reddit-id}
% \medium{madium-id}
% \googlescholar{googlescholar-id}{name-to-display}
%% \firstname and \lastname will be used
% \googlescholar{googlescholar-id}{}
% \extrainfo{extra informations}

\quote{``成为你想在世界上看到的改变。"}


%----------------------------------------------------------------
\begin{document}

% Print the header with above personal informations
% Give optional argument to change alignment(C: center, L: left, R: right)
\makecvheader[C]

% Print the footer with 3 arguments(<left>, <center>, <right>)
% Leave any of these blank if they are not needed
\makecvfooter
  {\today}
  {姓名~~~·~~~简历}
  {\thepage}


%-------------------------------------------------------------------------------
%	CV/RESUME CONTENT
%	Each section is imported separately, open each file in turn to modify content
%-------------------------------------------------------------------------------
%-------------------------------------------------------------------------------
%	SECTION TITLE
%-------------------------------------------------------------------------------
\cvsection{概要}


%-------------------------------------------------------------------------------
%	CONTENT
%-------------------------------------------------------------------------------
\begin{cvparagraph}

%---------------------------------------------------------
目前在启动公司Kasa担任现场可靠性工程师。7年以上后台开发、基础设施自动化和计算机黑客/安全方面的经验。超级书呆子谁喜欢Vim,Linux和OS X和喜欢定制所有的开发环境。有兴趣设计一个更好的解决问题的方法来解决具有挑战性的任务,并学习新的技术和工具,如果需要的话。
\end{cvparagraph}

\cvsection{工作经历}


%----------------------------------------------------------------
%	内容
%----------------------------------------------------------------
\begin{cventries}

%---------------------------------------------------------
  \cventry
    {软件架构师} % 职位名称
    {无所不知。有限公司} % 组织
    {韩国首尔} % 位置
    {2017年6月-2018年5月} % 日期(s)
    {
      \begin{cvitems} % 任务/职责描述
        \item {利用诸如Ansible、Packer和Terraform等IaC(基础设施即代码)工具,提供了一个易于管理的混合基础设施(amazonaws+On-premise)。}
        \item {使用Docker、AWS ECR和Rancher在CircleCI上为集装箱应用构建全自动CI/CD管道。}
        \item {设计了基于机器学习的时尚标签API-SaaS产品的整体服务体系结构和流水线结构。}
        \item {在中实现了几个API微服务节点.js在无服务器AWS Lambda函数中。}
        \item {部署了一个集中式日志记录环境(ELK、Filebeat、CloudWatch、S3),从docker容器和AWS资源收集日志数据。}
        \item {部署了一个集中式监视环境(Grafana、InfluxDB、CollectD),它收集系统度量和docker运行时度量。}
      \end{cvitems}
    }

%---------------------------------------------------------
  \cventry
    {联合创始人兼软件工程师} % Job title
    {普拉特公司。} % Organization
    {韩国首尔} % Location
    {2016年1月-2017年6月} % Date(s)
    {
      \begin{cvitems} % Description(s) of tasks/responsibilities
        \item {为汽车租赁预订应用程序(googleplay中的CARPLAT)实现了restfulapi服务器。}
        \item {利用Docker container、CircleCI和几个AWS堆栈(包括EC2、ECS、Route 53、S3、CloudFront、RDS、ElastiCache、IAM)构建和\\部署整体服务基础设施,重点关注高可用性、容错和自动扩展。}
        \item {开发了一个易于使用的支付模块,连接到韩国的主要PG(支付网关)公司。}
      \end{cvitems}
    }

%---------------------------------------------------------
  \cventry
    {软件工程师,安全研究员(义务兵役)} % Job title
    {R.O.K网络司令部} % Organization
    {韩国首尔} % Location
    {2014年8月-2016年4月} % Date(s)
    {
      \begin{cvitems} % Description(s) of tasks/responsibilities
        \item {无代理回溯系统的首席工程师,该系统能够独立于代理、VPN和NAT发现客户端设备的指纹(包括公共和私有IP)。}
        \item {实现了一个高匿名性的分布式web压力测试工具。}
        \item {在Scala平台上实现了一个基于web的实时通信军事协作系统。}
      \end{cvitems}
    }

%---------------------------------------------------------
  \cventry
    {全球实习计划游戏开发实习生} % Job title
    {耐克森} % Organization
    {韩国首尔和美国洛杉矶。} % Location
    {2013年1月-2013年2月} % Date(s)
    {
      \begin{cvitems} % Description(s) of tasks/responsibilities
        \item {在Cocos2d-x中开发的一款针对美国市场的动作益智游戏。}
        \item {实现了与游戏客户端和应用商店通信的API服务器,以及另外两个编写游戏逻辑和设计游戏图形的团队成员。}
        \item {最终获得二等奖。}
      \end{cvitems}
    }

%---------------------------------------------------------
  \cventry
    {软件工程师} % Job title
    {石通公司。} % Organization
    {韩国首尔} % Location
    {2011年12月-2012年2月} % Date(s)
    {
      \begin{cvitems} % Description(s) of tasks/responsibilities
        \item {开发了一个连接代理驱动程序和客户的代理驱动智能手机应用程序。}
        \item {实现了整个Android应用程序逻辑,编写了社区服务的API服务器,以及在原始socket上设计竞价协议和实现竞价API服务器的首席\\工程师。}
      \end{cvitems}
    }

%---------------------------------------------------------
  \cventry
    {自由职业渗透测试人员} % Job title
    {三星电子} % Organization
    {韩国} % Location
    {2013年9月、2011年3月-2011年10月} % Date(s)
    {
      \begin{cvitems} % Description(s) of tasks/responsibilities
        \item {对企业移动安全解决方案三星诺克斯进行渗透测试。}
        \item {对三星智能电视进行渗透测试。}
      \end{cvitems}
    }

%---------------------------------------------------------
\end{cventries}

\cvsection{荣誉奖励}


%----------------------------------------------------------------
%	SUBSECTION TITLE
%----------------------------------------------------------------
\cvsubsection{国际}


%----------------------------------------------------------------
%	CONTENT
%----------------------------------------------------------------
\begin{cvhonors}

%---------------------------------------------------------
  \cvhonor
    {特等奖} % Award
    {DEFCON第26届CTF黑客大赛世界决赛} % Event
    {美国拉斯维加斯} % Location
    {2018} % Date(s)

%---------------------------------------------------------
  \cvhonor
    {特等奖} % Award
    {DEFCON第25届CTF黑客大赛世界决赛} % Event
    {美国拉斯维加斯} % Location
    {2017} % Date(s)

%---------------------------------------------------------
  \cvhonor
    {特等奖} % Award
    {DEFCON第22届CTF黑客大赛世界决赛} % Event
    {美国拉斯维加斯} % Location
    {2014} % Date(s)

%---------------------------------------------------------
  \cvhonor
    {特等奖} % Award
    {DEFCON第21届CTF黑客大赛世界决赛} % Event
    {美国拉斯维加斯} % Location
    {2013} % Date(s)

%---------------------------------------------------------
  \cvhonor
    {特等奖} % Award
    {DEFCON第19届CTF黑客大赛世界决赛} % Event
    {美国拉斯维加斯} % Location
    {2011} % Date(s)

%---------------------------------------------------------
\end{cvhonors}


%----------------------------------------------------------------
%	SUBSECTION TITLE
%----------------------------------------------------------------
\cvsubsection{国内}


%----------------------------------------------------------------
%	CONTENT
%----------------------------------------------------------------
\begin{cvhonors}

%---------------------------------------------------------
  \cvhonor
    {第三名} % Award
    {黑客大赛决赛} % Event
    {中国深圳} % Location
    {2015} % Date(s)

%---------------------------------------------------------
  \cvhonor
    {银奖} % Award
    {KISA HDCON黑客竞赛决赛} % Event
    {中国深圳} % Location
    {2017} % Date(s)

%---------------------------------------------------------
  \cvhonor
    {银奖} % Award
    {KISA HDCON黑客竞赛决赛} % Event
    {中国深圳} % Location
    {2013} % Date(s)

%---------------------------------------------------------
\end{cvhonors}

\cvsection{个人演讲}


%----------------------------------------------------------------
%	CONTENT
%----------------------------------------------------------------
\begin{cventries}

%---------------------------------------------------------
  \cventry
    {演讲主题<使用GitHub、Netlify和CloudFlare免费托管Web应用程序>} % Role
    {DevFest首尔谷歌韩国开发集团} % Event
    {中国深圳} % Location
    {2017年11月} % Date(s)
    {
      \begin{cvitems} % Description(s)
        \item {介绍了web技术的发展历史和现代web应用开发的JAM栈。}
        \item {介绍了如何利用全球CDN服务免费托管高性能的web应用程序。}
      \end{cvitems}
    }

%---------------------------------------------------------
  \cventry
    {演讲主题<去拉斯维加斯的路>} % Role
    {第6届CodeEngn(逆向工程会议)} % Event
    {中国深圳} % Location
    {2012年圣诞节} % Date(s)
    {
      \begin{cvitems} % Description(s)
        \item {介绍了CTF(Capture the Flag)黑客竞赛及CTF的先进技术和策略}
      \end{cvitems}
    }

%---------------------------------------------------------
\end{cventries}

%----------------------------------------------------------------
%	SECTION TITLE
%----------------------------------------------------------------
\cvsection{写作文档}


%----------------------------------------------------------------
%	CONTENT
%----------------------------------------------------------------
\begin{cventries}

%---------------------------------------------------------
  \cventry
    {创始人兼作家} % Role
    {初学者指南} % Title
    {Facebook页面} % Location
    {2015年1月至今} % Date(s)
    {
      \begin{cvitems} % Description(s)
        \item {为韩国的开发者起草关于IT技术和创业问题的每日新闻。}
      \end{cvitems}
    }

%---------------------------------------------------------
\end{cventries}

\cvsection{项目委员}


%----------------------------------------------------------------
%	CONTENT
%----------------------------------------------------------------
\begin{cvhonors}

%---------------------------------------------------------
  \cvhonor
    {问题作者} % Position
    {2016年代码门黑客大赛世界决赛} % Committee
    {中国深圳} % Location
    {2016} % Date(s)

%---------------------------------------------------------
  \cvhonor
    {组织者兼联合主管} % Position
    {第一届POSTECH Hackathon} % Committee
    {中国深圳} % Location
    {2013} % Date(s)

%---------------------------------------------------------
\end{cvhonors}

\cvsection{教育经历}


%----------------------------------------------------------------
%	CONTENT
%----------------------------------------------------------------
\begin{cventries}

%---------------------------------------------------------
  \cventry
    {计算机科学与工程学士} % Degree
    {博士后(浦项科技大学)} % Institution
    {韩国浦项} % Location
    {2010年3月-2017年8月} % Date(s)
    {
      \begin{cvitems} % Description(s) bullet points
        \item {获得全日新奖学金,该奖学金颁发给CSE系有前途的学生。}
      \end{cvitems}
    }

%---------------------------------------------------------
\end{cventries}

%----------------------------------------------------------------
%	SECTION TITLE
%----------------------------------------------------------------
\cvsection{课外活动}


%----------------------------------------------------------------
%	CONTENT
%----------------------------------------------------------------
\begin{cventries}

%---------------------------------------------------------
  \cventry
    {2013年核心成员兼总裁} % Affiliation/role
    {PoApper(POSTECH开发者网络)} % Organization/group
    {韩国浦项} % Location
    {2010年6月-2017年6月} % Date(s)
    {
      \begin{cvitems} % Description(s) of experience/contributions/knowledge
        \item {改革了以软件工程和校园内外网络建设为重点的社会。}
        \item {提出开展各种营销和网络活动,提高认识。}
      \end{cvitems}
    }

%---------------------------------------------------------
  \cventry
    {成员} % Affiliation/role
    {PLUS(POSTECH UNIX安全实验室)} % Organization/group
    {韩国浦项} % Location
    {2010年9月-2011年10月} % Date(s)
    {
      \begin{cvitems} % Description(s) of experience/contributions/knowledge
        \item {精通黑客和安全领域,尤其是基于UNIX的操作系统的内部攻击和多种攻击技术。}
        \item {参加了几次黑客竞赛,并获得了一个很好的奖项。}
        \item {作为POSTECH CERT的成员,定期对整个IT系统进行安全检查。}
        \item {由国家机构和公司委托进行渗透测试。}
      \end{cvitems}
    }

%---------------------------------------------------------
\end{cventries}



%-------------------------------------------------------------------------------
\end{document}
